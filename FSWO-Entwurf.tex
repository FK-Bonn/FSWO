% !TeX program = lualatex

\documentclass[%
draft,%
%sectionprefix=SECTIONPREFIX,% Uncomment to change the section prefix
multilinesections%
]{fswo}

\hypersetup{pdfinfo={%                  % Zusatzinformationen in PDF-Datei;
                                        % alle Werte sind optional.
    Author={Fachschaftenkonferenz Bonn},
    CreationDate={D:2020MMTT181500},    % Datum der Erstellung %todo
                                        % (D:JJJJMMTThhmmss)
                                        % JJJJ  Jahr
                                        % MM    Monat
                                        % TT    Tag
                                        % hh    Stunden
                                        % mm    Minuten
                                        % ss    Sekunden
                                        %
                                        % Standard: Das aktuelle Datum
                                        %
%   ModDate={D:20011010111111},         % Datum der letzten Modifikation
    Title={Wahlordnung für die Wahlen der Fachschaftsvertretungen und Fachschaftsräte (Fachschaftswahlordnung - FSWO)},
    Subject={Wahlordnung},
    Keywords={FSWO, FK, Wahlordnung, Fachschaftswahlordnung, Fachschaft, Bonn}
}}


% Mit \oldT werden gestrichene Passagen markiert und mit \bemFr werden Bemerkungen von Franz eingefügt.
% Beides wird in der fertigen Version nicht angezeigt.
\makeatletter
  \ifFK@draft% Version mit Kommentaren, die Absätze mit Kommentaren sind zusätzlich kursiv
    \newcommand\oldT[1]  {{\color{Gray}[\st{#1}]}}
    \newcommand\newT[1]  {{\color{Green}[#1]}}
    \newcommand\bemFr[1] {{\color{Red}[#1]}}
    \newcommand\bemFe[1] {{\color{Cyan}[#1]}}
  \else %Fertige Version
    \newcommand\oldT[1]{}% streichen
    \newcommand\newT[1]{#1}
    \newcommand\bemFr[1]{}% streichen
    \newcommand\bemFe[1]{}% streichen
  \fi
\makeatother

\newcommand\change[2]{\oldT{#1}\newT{#2}}

\begin{document}
%\subject{Wahlordnung}
\title{Wahlordnung für die Wahlen der Fachschaftsvertretungen und Fachschaftsräte\\
(Fachschaftswahlordnung -- FSWO)}
\subtitle{Rheinische Friedrich-Wilhelms-Universität Bonn}
\author{}
\date{{\itshape\normalsize%
  Beschlossen am dd.\,mm.\,2020 durch die Fachschaftenkonferenz}}
\publishers{\itshape\normalsize%
  sowie am dd.\,mm.\,2020 durch das Studierendenparlament.}

\maketitle

\bemFr{\textbf{Anmerkungen zum Titel}: Dass diese Ordnung nur für Organe der Studierendenschaft der Universität Bonn gilt, ergibt sich bereits aus \S 1 dieser Ordnung (Geltungsbereich), die Streichung kürzt den Titel, der so weniger sperrig wirkt. Zusätzlich werden mit \enquote{Fachschaftswahlordnung} eine rechtlich eindeutige Kurzbezeichnung und mit \enquote{FSWO} eine rechtlich eindeutige Abkürzung eingeführt.}

\begingroup
%% Remove comment to remove colours of the links in the table of contents
%    \hypersetup{linkcolor=black}
%    \tableofcontents
\endgroup


\bemFr{\textbf{zur Zwischenüberschrift:} Die FSWO wird durch Einfügung von Abschnittsüberschriften gegliedert.}
\section{ALLGEMEINE BESTIMMUNGEN}

\begin{contract}
\Clause{title={Geltungsbereich}}
Diese Wahlordnung gilt für die Wahl zu den Fachschaftsvertretungen (FSV) und den Fachschaftsräten (FSR) der Universität Bonn.
\end{contract}

\bemFr{\textbf{Anmerkungen zu \S~2 neu:}
Im Folgenden wird \S~4 alt zu \S~2 neu, d. h. die Wahlgrundsätze werden nach vorn gezogen.
Das dient der Übersichtlichkeit und billigt den Wahlgrundsätze eine gewisse Priorität zu.
Weiterhin entfällt Abs. 1 S. 2 (bislang \S~4 I 2), da dieser in \S~3 neu aufgeht, vgl. Anmerkungen vor \S~3.}

\begin{contract}
\Clause{title={Wahlgrundsätze}}
Das zu wählende Organ wird von den Mitgliedern der Fachschaft jährlich in allgemeiner, unmittelbarer, freier, gleicher und geheimer Wahl gewählt. \oldT{Die Zahl der zu wählenden Mitglieder ergibt sich aus der Satzung der Studierendenschaft iVm der Fachschaftssatzung (siehe auch \S\S~2 und 3 FSWO).}

Die Wahl erfolgt als Persönlichkeitswahl.
\end{contract}

\bemFr{\textbf{Anmerkungen vor \S~3:} In der derzeitigen FSWO wird in dem Paragraphen \enquote{Direkte 38 und indirekte Wahl des Fachschaftsrates} auch die Wahl der Fachschaftsvertretung behandelt und auch sonst ist dieser Aufbau sehr unübersichtlich und hat in der Vergangenheit zu Problemen geführt.

Der erste Änderungsvorschlag der FK sah vor, hieraus jeweils einen Paragraphen zur Wahl der FSV und zur Wahl des FSR zu machen, dabei wurde jedoch die direkte Wahl eines FSR in Abs. 3 des Paragraphen zur Wahl der FSV behandelt, zudem wurden immer noch Gegenstand und Art der Wahl vermischt.
Für einen Leser ohne Vorkenntnisse wird so nicht klar, welche Fachschaft auf welche Weise welche Gremien zu wählen hat.

Daher sieht dieser Entwurf folgende Gliederung vor:
\S~3 Gegenstand der Wahl (Welche Fachschaft wählt welche Gremien auf welche Weise und wo muss man jeweils weiterlesen?), \S~4 Wahl einer Fachschaftsvertretung und \S~5 Wahl eines Fachschaftsrates (mit Details zum jeweiligen Gremium).

Zusätzlich wird ein \S~3a (jetzt 4) Amtszeit eingeführt und somit \S~6 alt nach vorn gezogen. In der alten Fassung stand die Amtszeit mitten im Prozedere der Wahlen selbst, dabei gehört die Amtszeit doch thematisch zum zu wählenden Organ. Auch dies dient also der Übersichtlichkeit und Lesbarkeit.}

\begin{contract}
\Clause{title={Gegenstand der Wahl}}
Fachschaften mit bis zu 500 Mitgliedern wählen nach \S~5 dieser Ordnung in direkter Wahl einen Fachschaftsrat (FSR)\newT{, sofern die Fachschaftssatzung nicht die Wahl einer FSV vorsieht.}
Fachschaften mit 501 oder mehr Mitgliedern wählen nach \S~4 dieser Ordnung in direkter Wahl eine Fachschaftsvertretung (FSV), die sodann einen FSR wählt (indirekte Wahl).

Die Mitgliederzahl der Fachschaft richtet sich nach dem Wählendenverzeichnis.
Die \S\S~6ff. dieser Ordnung betreffen die jeweilige direkte Wahl.
\end{contract}

\bemFr{\textbf{zu \S~3 neu:} In diesem Kontext stand bisher ein Verweis in die \S\S~26f. SdS – der wurde verworfen, da eine Umnummerierung der Satzung dann einen (wegen der nötigen Zustimmung zweier Gremien deutlich aufwendigeren) Änderungsprozess dieser Ordnung in Gang setzen würde.}

\bemFe{Der Absatz 3a wird zu Absatz 4.}

\begin{contract}
\Clause{title={Amtszeit}}%, number=3a}
Die Amtszeit der Mitglieder der Organe der Fachschaft \oldT{beträgt maximal ein} beträgt grundsätzlich ein Jahr und endet mit der Wahl von \change{Nachfolgern}{Nachfolgen}.
Nach Ablauf der Amtszeit bleiben die Mitglieder bis zur Wahl von \change{Nachfolgern}{Nachfolgen} kommissarisch im Amt.
\end{contract}

\bemFr{Die Amtszeit kann je nach Wahltermin auch geringfügig länger als ein Jahr dauern, ersetze daher \enquote{maximal} in \S~6 alt durch \enquote{grundsätzlich}.}

\begin{contract}
\Clause{title={Wahl einer Fachschaftsvertretung}}
Sofern eine FSV zu wählen ist, richtet sich ihre Größe nach der Größe der Fachschaft.
Die Anzahl der Mitglieder beträgt in Fachschaften
\begin{enumerate}
  \item mit 501 bis 1.000 Mitgliedern 11,
  \item mit 1.001 bis 2.000 Mitgliedern 15 und
  \item mit 2.001 oder mehr Mitgliedern 19.
\end{enumerate}

In Fachschaften mit bis zu 500 Mitgliedern kann die Fachschaftssatzung eine Regelung über die Einrichtung einer FSV treffen.
In diesem Fall beträgt die Anzahl ihrer Mitglieder 7.

\Clause{title={Wahl des Fachschaftsrates}}
Sofern der FSR nicht direkt zu wählen ist, erfolgt die Wahl durch die FSV.
Näheres regelt die Fachschaftssatzung vorbehaltlich der \S\S~20f. dieser Ordnung und der SdS.


\bemFe{Dieser Absatz wurde angepasst:}
%Der FSR besteht aus:
%\begin{enumerate}
%  \item der Vorsitzenden,
%  \item der stellvertretenden Vorsitzenden,
%  \item der Finanzreferentin,
%  \item bis zu 6 weiteren Mitgliedern, vorbehaltlich anderweitiger Regelung durch die Fachschaftssatzung sind dies 2,
%  \item sowie darüber hinaus bis zu 2 zusätzlichen Mitgliedern für jede Fach-Abschluss-Kombination im Sinne der SdS, sofern die Fachschaftssatzung diese vorsieht.
%Damit sind im Regelfall 5 Mitglieder zu wählen, sofern die Fachschaftssatzung nicht zusätzliche Mitglieder im Sinne der Nr. 4 und 5 vorsieht.
Der FSR besteht aus:
\begin{enumerate}
  \item der Vorsitzenden,
  \item der stellvertretenden Vorsitzenden,
  \item der Finanzreferentin,
  \item 2 bis 6 weiteren Mitgliedern, sowie
  \item gegebenenfalls zusätzlichen Mitgliedern gemäß Absatz 3.
\end{enumerate}
Falls die Fachschaftssatzung zu der Anzahl nach Nr. 4 keine Regelung trifft, sind zwei Mitglieder nach Nr. 4 zu wählen.

\newT{Sind in einer Fachschaft mehrere Fach-Abschluss-Kombinationen (FAKs) zusammengefasst und wird eine Fachschaftsvertretung gewählt, so kann diese gemäß \S~27 Abs. 5 SdS für jede dieser FAKs bis zu zwei zusätzliche Personen in den FSR wählen, die einen Studiengang mit dieser FAK studieren.}
\end{contract}

\bemFr{\textbf{Anmerkung zu \S~6 neu:} Zunächst ist \S~6 alt (Amtszeit) als \S~3a neu hochgerutscht ist, sodann sind wir überein gekommen, dass es unsinnig ist, die Kosten der Wahl vor dem Wahlsystem zu regeln. Damit wird \S~8 alt (Wahlsystem) zu \S~6 neu. Es wird hieran ein Abs. 6 angehängt, der die Durchführung als Brief- oder Urnenwahl betrifft. Er entspricht dem \S~9 I des vorigen Entwurfs einer FSWO-ÄO (darin Nr. 6), gehört aber thematisch zu Wahlsystem (vgl. auch Anmerkung zu \S~9 neu, Rn. 136ff.}

\begin{contract}
\Clause{title={Wahlsystem}}
Eine Fachschaft bildet einen Wahlkreis.
Jede wahlberechtigte Person hat eine Stimme, die sie für eine kandidierende Person abgibt.

Gewählt sind die Personen mit den meisten Stimmen, die mindestens 1 Stimme erhalten haben.

Bei Stimmgleichheit entscheidet die Wahlleitung durch Los über die Reihenfolge.

Scheidet ein gewähltes Mitglied aus, so wird der Sitz derjenigen kandidierenden Person zugeteilt, die nach dem Wahlergebnis unter den bisher nicht berücksichtigten Personen die meisten Stimmen, jedoch mindestens eine Stimme, hat.
Ist die Liste dieser Personen erschöpft, so bleibt der Sitz unbesetzt.

Ist zu einem Zeitpunkt mindestens die Hälfte der Sitze des Organs unbesetzt, finden innerhalb der nach dieser Wahlordnung kürzestmöglichen Zeit Neuwahlen statt.
D\oldT{er oder d}ie FSR-Vorsitzende beruft dann unverzüglich eine FSVV ein, auf der der Wahltermin festgelegt und ein Wahlausschuss gewählt wird. \bemFr{Anmerkung: Einhaltung des generischen Femininums in S. 2}

Die Wahl erfolgt als Urnenwahl.
\change{Die Briefwahl einzelner Wahlberechtigter}{Die Stimmabgabe durch Briefwahl} ist nach den Regeln \S~16 dieser Ordnung möglich.
\end{contract}

\bemFr{\S 7 bleibt gleich}
\begin{contract}
\Clause{title={Kosten}}
Alle der Fachschaft in Durchführung der Wahlen nach dieser Wahlordnung entstehenden Kosten werden aus ihrem Haushalt getragen.
\end{contract}

\bemFr{\textbf{Anmerkung zu \S~8 neu:} Da \S~8 alt zu \S~6 neu geworden ist (vgl. Rn. 93ff.), wird \S~9 118 alt zu \S~8 neu. In Abs. 1 S. 1 wird durch Einfügung von \enquote{Stichtag} der dreißigste Tag vor dem ersten Wahltag als Stichtag legaldefiniert. Abs. 2 alt wird ersatzlos gestrichen, denn der ist teils widersprüchlich, teils redundant:
Nachdem in Abs. 1 die Wahlberechtigung an den 30. Tag vor dem ersten Wahltag geknüpft wurde, trifft Abs. 2 S. 1 alt eine dem widersprechende Regelung, die an den Zeitpunkt der Wahl selbst anknüpft -- eine Ordnung darf aber keine sich widersprechenden Regelungen beinhalten.
Sodann wiederholt Abs. 2 S. 2 alt das, was in Abs. 1 bereits festgelegt wurde.
Deshalb entfällt Abs. 2 ersatzlos. In Abs. 3 alt bzw. Abs. 2 neu ist der Begriff \enquote{Studentensekretariat} durch den Begriff \enquote{Studierendensekretariat} zu ersetzen, da dieses inzwischen so heißt.}

\section{DURCHFÜHRUNG DER WAHL}

\begin{contract}
\Clause{title={Wahlberechtigung}}
Wahlberechtigt und wählbar sind die \change{Mitglieder der jeweiligen Fachschaft}{Personen}, die am 30. Tag vor dem ersten Wahltag (Stichtag) Mitglied der Fachschaft sind.
Maßgeblich für die Wahlberechtigung ist darüber hinaus die für die Wahlberechtigung angegebene Fach-Abschluss-Kombination (FAK).
Zweit- und Gasthörerinnen\oldT{ und -hörer} sind nicht wahlberechtigt.

Die Wahlberechtigung ist im Studierendenausweis vermerkt.
Die Änderung der Wahlberechtigung ist im Studierendensekretariat der Universität möglich.
\end{contract}

\bemFr{\textbf{Anmerkungen zu \S~9 neu:}
\S~9 neu entspricht \S~9 des vorigen FSWO-ÄO-Entwurfs (Nr. 6), es entfällt jedoch der erste Absatz, da er thematisch nicht zum Wahltermin, sondern zum Wahlsystem gehört
und deshalb als \S~6 VI neu Aufnahme gefunden hat, vgl. Rn. 95ff.;
weiterhin wird aus sprachlich-ästhetischen Gründen und um den Gleichlauf der Fristen zu betonen in Abs. 3 neu ein \enquote{Ebenfalls} eingefügt.}

\begin{contract}
\Clause{title={Festlegung des Wahltermins}}
Gewählt wird an mindestens drei aufeinander folgenden nicht vorlesungsfreien Werktagen (Montag -- Freitag).

Die FSV bestimmt spätestens bis zum 30. Tag vor dem 1. Wahltag den Termin und die Dauer der Wahl.
Falls keine FSV besteht, übernimmt diese Aufgabe der FSR.

\oldT{Ebenfalls bis zum 30. Tag vor dem ersten Wahltag wählt die Fachschaftsvertretung den Wahlausschuss in öffentlicher Sitzung und benachrichtigt die Gewählten und das Fachschaftenkollektiv.
Falls keine FSV besteht, übernimmt diese Aufgabe der FSR.
Die Wahl von Wahlleitung und Wahlausschuss muss in der Sitzungseinladung angekündigt werden.}
\end{contract}

\bemFr{\textbf{Anmerkungen zu \S~10 neu:}
Bis dato mussten Aufgaben, die die FSWO der Wahlleitung (ohne nähere Spezifizierung) zuweist, von beiden Mitgliedern übernommen werden, also von der Wahlleiterin und deren Stellvertreterin.
Es stehen drei Vorschläge zur Lösung dieses Problems im Raum:

Nr. 1 (Vorschlag des vorherigen ÄO-Entwurfs):
Das Amt der Stellvertreterin entfällt, die Wahlleitung besteht nur noch aus einer Person.

Nr. 2 (Vorschlag Franz Janßen):
Das Organ \enquote{Wahlleitung} wird ersetzt durch das Organ 157 \enquote{Wahlleiterin}.
Es wird aus den weiteren Mitgliedern des Wahlausschusses eine personelle Stellvertreterin gewählt,
die aber wirklich nur Aufgaben und Rechte der Wahlleiterin in deren Absenz wahrnimmt, also nicht Teil der Wahlleitung ist.

Nr. 3 (Vorschlag Marlon Brüßel):
Es wird ein \S~10 IX angefügt: \enquote{Aufgaben der Wahlleitung können nach Absprache auch von ihren einzelnen Mitgliedern wahrgenommen werden.}

\textbf{Persönliche Anmerkung:} Ich (Franz, hi) möchte an dieser Stelle für Marlons Vorschlag plädieren.
Mein eigener Vorschlag erscheint mir inzwischen recht unsinnig –- es besteht kein Grund,
die Rechte und Pflichten der Stellvertreterin auf Fälle der Absenz der Amtsinhaberin zu beschränken, im Gegenteil:
Unser Ziel ist ja eine effizientere Arbeitsteilung und nicht gar keine Arbeitsteilung mehr.
Der Vorschlag Nr. 1 hat ein ähnliches Manko, warum sollten Fachschafts-Wahlausschüsse auf eine Stellvertreterin verzichten?
Zudem ist Vorschlag Nr. 3 am unbürokratischsten, seine Umsetzung bedarf lediglich der Einfügung des Abs. 9 (vgl. Rn. 220ff.),
zur Umsetzung meines Vorschlags dagegen müsste in der ganzen Ordnung der Begriff \enquote{Wahlleitung} durch den Begriff \enquote{Wahlleiterin} ersetzt werden.}

\begin{contract}
\Clause{title={Wahlorgane}}
Der Wahlausschuss besteht aus der Wahlleiterin (im Folgenden \enquote{die Wahlleitung}) und mindestens zwei weiteren Mitgliedern.
Mitglieder des Wahlausschusses dürfen für die Wahl nicht kandidieren.
\change{Die Wahlleitung soll der Fachschaft angehören.}{Die Mitglieder des Wahlausschusses müssen der Studierendenschaft angehören.}
%
\bemFr{%
\begin{addmargin}{1cm}
Das ist die Formulierung, die Vorschlag Nr. 1 entspricht.
Um Vorschlag Nr. 3 zu entsprechen, müsste es heißen: \enquote{besteht aus der Wahlleiterin und einer stellvertretenden Wahlleiterin (im Folgenden \enquote{die Wahlleitung})}.

Zudem: Abs. 2 entfällt, wie im vorigen Entwurf vorgeschlagen, da obsolet, vgl. \S~9 III neu, Rn. 146; hier folgt Abs. 2 neu:
\end{addmargin}}

\change{Bis zum Stichtag wählt die FSV die Wahlleiterin, die übrigen Mitglieder des Wahlausschusses und aus deren Mitte eine stellvertretende Wahlleiterin in öffentlicher Sitzung.
Die Wahl der jeweiligen Person ist wirksam, nachdem diese ihr zugestimmt hat.
Die Wahl der Wahlorgane muss in der Sitzungseinladung der FSV angekündigt werden.
Die FSV sollte das Fachschaftenkollektiv über die Gewählten benachrichtigen.}{%
Bis zum 30. Tag vor dem ersten Wahltag wählt die FSV die Wahlleiterin und die übrigen Mitglieder des Wahlausschusses in öffentlicher Sitzung, zu der öffentlich eingeladen werden muss.
Falls keine FSV besteht, übernimmt diese Aufgabe der FSR.
Die Wahl des Wahlauschusses muss in der Sitzungseinladung angekündigt werden.
Die Wahl der jeweiligen Person ist wirksam, nachdem diese ihr zugestimmt hat.}\\
%
\begin{addmargin}{1cm}
\bemFr{\textbf{Anmerkung zu Abs. 2a neu:} Wer die Wahlorgane bestimmt, wenn kein FSV besteht,
war in der alten Wahlordnung recht \enquote{versteckt} mitten im Absatz über das Verfahren im Regelfall.
Das soll nun etwas prominenter hervorgehoben werden, daher der eigene Absatz:}
\end{addmargin}
%
\oldT{(2a) Wo keine FSV besteht, übernimmt der FSR die Aufgaben aus Absatz 2.}
%\bemFe{Hier werden aus technischen Gründen erstmal die Absätze konsekutiv ohne Einschub gesetzt. In Franz' Enwurf war dies abweichend Absatz 2a.}


\begin{addmargin}{1cm}
\bemFr{Abs. 3 neu wird aus dem vorigen Entwurf für eine FSWO-ÄO übernommen, es wird lediglich in S. 4 \enquote{Amt} durch \enquote{Aufgaben} ersetzt und die Formulierung an das generische Femininum angepasst,
zudem hatte sich ein kleiner Fehlerteufel eingeschlichen:
In S. 4, 1. Hs. war von der Wahlleitung die Rede, wo wohl das zurückgetretene WA-Mitglied gemeint war.}
\end{addmargin}

Mitglieder des Wahlausschusses können jederzeit zurücktreten.
Sie sind verpflichtet, ihr Amt bis zur Wahl einer Nachfolgerin kommissarisch weiterzuführen.
Im Falle eines Rücktritts ist durch die FSV bzw. den FSR zum frühestmöglichen Zeitpunkt eine Nachfolgerin zu wählen.
\change{Ist es dem zurückgetretenen Mitglied des Wahlausschusses nicht möglich, seine Aufgaben bis zur Wahl einer Nachfolge im Sinne dieser Ordnung auszuführen, kann es einmalig seine Aufgaben für diesen Zeitraum an ein anderes Mitglied übertragen.}{%
Ist es einer zurückgetretenen Wahlleiterin nicht möglich, ihr Amt bis zur Wahl einer
Nachfolge im Sinne dieser Ordnung auszuführen, kann sie einmalig ihr Amt für diesen
Zeitraum an ein anderes Mitglied übertragen.}
Über diesen Vorgang ist ein Protokoll zu führen.
%
\\\bemFe{Die Absätze alt 4 bis 6 wurden an den Anfang von Paragraphen neu 13 verschoben.}

Die Wahlleitung sichert in Abstimmung mit dem Fachschaftenkollektiv und der Hochschulverwaltung die technische Vorbereitung und Durchführung der Wahl.
Sie führt die Beschlüsse des Wahlausschusses aus.
%\bemFr{unverändert}

Der Wahlausschuss entscheidet bei Streitigkeiten nach Rücksprache mit dem Fachschaftenkollektiv über die Auslegung der Wahlordnung.
%\bemFr{unverändert}
%
\begin{addmargin}{1cm}
\bemFr{\textbf{Anmerkung:} Sofern wir uns bzgl. Wahlleitung (vgl. Anmerkungen vor \S~10, Rn. 151ff.) für Vorschlag Nr. 3
222 entscheiden, muss hier der Abs. 9 zur Aufgabenverteilung innerhalb der Wahlleitung eingefügt werden.}
\end{addmargin}

\change{Aufgaben der Wahlleitung können nach Absprache auch von ihren einzelnen Mitgliedern wahrgenommen werden.}{%
Die Wahlleitung kann einzelne ihrer Aufgaben nach Absprache auch anderen Wahlausschussmitgliedern übertragen. Über diesen Vorgang ist ein Protokoll zu führen.}
%
\begin{addmargin}{1cm}
\bemFr{\textbf{Anmerkung:} Der Vorschlag aus dem vorigen Entwurf einer FSWO-ÄO, einen Absatz zu virtuellen Sitzungen anzufügen, wird als \S~11 VII umgesetzt.}
\end{addmargin}
\end{contract}

\bemFr{\textbf{Anmerkungen zu \S~10a neu:}
Die verschiedenen Anforderungen an Veröffentlichungen des Wahlausschusses werden im Folgenden in einer einzufügenden neuen Vorschrift vereinheitlicht und gebündelt.
Dies dient auch der Lesbarkeit und Verständlichkeit, zudem können etwaige Änderungen so zentral und weniger bürokratisch umgesetzt werden.}

\bemFe{10a wird zu 12}

\begin{contract}
\Clause{title={Veröffentlichungen des Wahlausschusses}}%, number=10a}
Öffentliche Bekanntmachungen des Wahlausschusses sind an angemessenen Stellen jeweils als Aushang fachschaftsöffentlich sowie im Internet zu veröffentlichen.
Der Aushang soll an ortsüblicher Stelle erfolgen.

\oldT{Öffentliche Bekanntmachungen erfolgen gemäß \S~10 Abs. 4, \S~12 Abs. 3, \S~13 Abs. 1, \S~16 Abs. 6 und \S~19 Abs. 5.} \bemFr{(AUF VOLLSTÄNDIGKEIT PRÜFEN; ENUMERATIONSPRINZIP!!! Hilfsweise \enquote{insbesondere} einfügen, dann ist entsprechende Anwendung möglich).}

Protokolle gemäß \S~10 Abs. 6 müssen nur \change{auf der Website}{im Internet} veröffentlicht werden.
Sie sind zudem auf Nachfrage einsehbar zu machen.

Veröffentlichungen im Internet \change{können auf der Homepage des FSR erfolgen.}{Veröffentlichungen im Internet sollen auf der Webpräsenz der Fachschaft erfolgen.}
Die Wahlleitung darf den FSR in diesem Fall mit der Veröffentlichung beauftragen, bleibt allerdings weiterhin für die Einhaltung dieser Ordnung verantwortlich.
\end{contract}

\bemFr{\textbf{Anmerkungen zu \S~11:}
Die Norm soll sich inhaltlich nicht verändern, es erfolgen jedoch einige Schönheitskorrekturen.
Zunächst werden die Absätze 1 und 2 zusammengezogen, da sie sich auf das gleiche Fristsetzungserfordernis zu Lasten des selben Organs beziehen.
Sodann wird mit Abs. 2 neu eine Möglichkeit eingeführt, die Beschlüsse nach Abs. 1 notfalls noch zu korrigieren,
wenn äußere Umstände (wie eine Pandemie) dies erfordern.

In Abs. 3 wird auf Rechtsinstitute dieser Ordnung verwiesen, ohne dass dies durch einen Normverweis kenntlich gemacht wird.
Dies wird hier geändert und verbessert die Verständlichkeit um Einiges.
In Abs. 3 Nr. 2 und Abs. 4 S. 1 wird der rechtlich korrekte Terminus der Nachfrist eingeführt.

Ein Absatz 5 wird eingeführt, der die Festlegung der Reihenfolge der Kandidatinnen auf dem Stimmzettel behandelt.
Ein Abs. 6 wird eingeführt, der einen Verweis auf Einladung,256 Beschlussfähigkeit, Öffentlichkeit der Sitzungen und Protokollführung enthält –
dies hat den Zweck, dass der eine Sitzung vorbereitende Leser dieser Ordnung auf diese Punkte auch dann hingewiesen wird,
wenn er nur die Norm zu Sitzungen liest. Im gleichen Absatz soll die Möglichkeit virtueller Sitzungen geregelt werden.}

\bemFe{Der erste Absatz wird in zwei Absätze aufgeteilt.}
\begin{contract}
\Clause{title={Sitzungen des Wahlausschusses}}
\begin{addmargin}{1cm}
\bemFr{in Abs. \change{4}{1} wird ein Verweis auf \change{\S~10a}{\S~12} eingefügt, siehe dortige Kommentierung.}\bemFe{Die Absätze neu 1 bis 3 wurden aus \S~neu 11 her verschoben.}
\end{addmargin}

Die Einladungen zu den Sitzungen des Wahlausschusses erfolgen durch öffentliche Bekanntmachung gem. \S~10a dieser Ordnung durch die Wahlleitung spätestens 24 Stunden vor Sitzungsbeginn.
Die weiteren Mitglieder des Wahlausschusses sind außerdem schriftlich oder per E-Mail zu laden.
Der Wahlausschuss kann eine andere Form für die Ladung beschließen.
Für die Wahlwoche kann der Wahlausschuss eine kürzere Frist beschließen.

Der Wahlausschuss ist beschlussfähig, wenn die Wahlleitung anwesend ist und zur Sitzung ordnungsgemäß \newT{und öffentlich} geladen wurde.
Die Sitzungen des Wahlausschusses sind öffentlich.
%
\begin{addmargin}{1cm}
\bemFr{\textbf{zu Abs. 6:}
Der Umgang mit Protokollen war bislang in der Ordnung ungeklärt,
dem mit der hochschulpolitischen Parlamentspraxis nicht vertraute Anwender hat keinen Anhaltspunkt, wie er mit den Protokollen zu verfahren hat.
Dies soll hier angepasst werden.]}
\end{addmargin}
%
\oldT{Der Wahlausschuss fertigt über seine Sitzungen Protokolle an, die die Wahlleitung unterzeichnet.}

Der Wahlausschuss fertigt über seine Sitzungen Protokolle an.
Am Ende der Sitzung oder auf der nächsten Sitzung ist das Protokoll auf seine Richtigkeit zu prüfen, gegebenenfalls anzupassen und per Beschluss als richtig zu verabschieden.
Danach ist das Protokoll durch die Wahlleitung zu unterzeichnen,
\change{gem. \S~10a Abs. 3 dieser Ordnung zu veröffentlichen und zumindest bis zur nächsten Wahl im Original zur Einsichtnahme aufzubewahren.
Nach seiner Konstituierung ist der Wahlausschuss für die Veröffentlichung und Aufbewahrung der Protokolle zuständig.}{%
gem. \S~10a Abs. 3 zu veröffentlichen und gem. \S~23 Abs. 6 dieser Ordnung durch den FSR im Original zur Einsichtnahme aufzubewahren.
Der Wahlauschuss ist für die Veröffentlichung der Protokolle zuständig.}

Bis zum 25. Tag vor dem ersten Wahltag findet die konstituierende Sitzung des Wahlausschusses statt.

\change{Bis zum gleichen Tag}{Bis zum 25. Tag vor dem ersten Wahltag} fasst der Wahlausschuss folgende Beschlüsse:
\begin{enumerate}
\item Festlegung der Urnenstandorte und -öffnungszeiten,
\item Festlegung der Auslageorte für das Wählendenverzeichnis und des Zeitraums der Auslage,
\item Festlegung der gemeinsamen Frist für die Einreichung von Kandidaturen, Briefwahlanträgen und Einsprüchen gegen das Wählendenverzeichnis,
\item Festlegung des Zeitpunkts des Endes der Wahl und des Ortes für die Auszählung der Wahl,
\item Festlegung des Ortes und Termins für die konstituierende Sitzung des gewählten Organs.
\end{enumerate}
Die Frist für die Einreichung von Kandidaturen, Briefwahlanträgen und Einsprüchen gegen das Wählendenverzeichnis darf frühestens auf dem 13. Tag und spätestens auf dem 10. Tag vor dem ersten Wahltag liegen.
Sie darf nicht auf einem vorlesungsfreien Tag liegen.

Der Wahlausschuss kann, sofern äußere Umstände dies erfordern, auf einer weiteren Sitzung Beschlüsse nach Absatz 2 mit Einstimmigkeit der anwesenden Mitglieder neu fassen.
Die Frist nach Absatz 2 Nr. 3 darf nicht auf diesem Weg \change{vorverlegt}{auf einen früheren Zeitpunkt verlegt} werden.
%TODO: Müsste das nich "späteren" heißen?

Eine weitere Sitzung des Wahlausschusses findet unmittelbar nach Ablauf der Frist für die Einreichung von Kandidaturen, Briefwahlanträgen und Einsprüchen gegen das Wählendenverzeichnis statt.
\change{Auf ihr fasst der Wahlausschuss nach Prüfung der eingereichten Kandidaturen folgende Beschlüsse:}{%
Auf ihr fasst der Wahlausschuss nach Prüfung der eingereichten Kandidaturen und gegebenenfalls Korrektur von vom Wahlausschuss selbst behebbaren Mängeln folgende Beschlüsse:}
\begin{enumerate}
\item Zulassung der mängelfreien Kandidaturen zur Wahl\oldT{ gem. \S~14 Abs. 3 S. 1 dieser Ordnung},
\item ggf. Feststellung von Mängeln in eingereichten Kandidaturen und Setzen einer angemessenen Nachfrist\oldT{ im Sinne von \S~14 Abs. 3 S. 2 dieser Ordnung},
\item ggf. Entscheidung über Anträge auf Briefwahl,
\item ggf. Entscheidung über Einsprüche gegen das Wählendenverzeichnis.
\end{enumerate}

Falls eine Nachfrist im Sinne von Abs. 3 Punkt 2 gesetzt wurde, findet eine weitere Sitzung des Wahlausschusses unverzüglich nach Ablauf dieser Frist statt.
Auf dieser Sitzung entscheidet der Wahlausschuss über die Zulassung der übrigen Kandidaturen.

\change{Nachdem über die Zulassung aller Kandidaturen entschieden wurde, ist die Reihenfolge, in der die Kandidatinnen gemäß \S~15 Abs. 2 dieser Ordnung auf dem Stimmzettel abgedruckt werden, durch den Wahlausschuss festzulegen.}{%
Nachdem über die Zulassung aller Kandidaturen entschieden wurde, ist die zufällige Reihenfolge, in der die Kandidaturen auf dem Stimmzettel abgedruckt werden, durch den Wahlausschuss auszulosen.}

Eine weitere Sitzung des Wahlausschusses findet zur Auszählung der Wahl statt. Auf ihr stellt der Wahlausschuss das Wahlergebnis fest.

Die Wahlleitung kann zu weiteren Sitzungen einladen.

\oldT{Bei jeder Sitzung des Ausschusses ist \S~10 Absätze 4 bis 6 dieser Ordnung einzuhalten.}
Die Wahlleitung kann festlegen, dass Sitzungen des Wahlausschusses als virtuelle Sitzung in elektronischer Kommunikation stattfinden\oldT{, solange höherrangige Rechtsquellen dies erlauben}.
Die Wahlleitung stellt sicher, dass die Öffentlichkeit auch an elektronischen Sitzungen teilnehmen kann.
\end{contract}

\bemFr{\textbf{zu \S~12:}
In Absatz 3 wird ein Normverweis hinsichtlich der Einsprüche gegen das Wählendenverzeichnis eingefügt.
In Absatz vier wird S. 2 eingefügt, der festlegen soll, dass Einsprüche begründet sein müssen
– diese Formulierung lässt mehr individuellen Spielraum, als die zwischenzeitlich angedachte Formulierung,
dass Einsprüche einen spezifischen Mangel am Wählendenverzeichnis ausdrücklich nennen müssen.
Zudem wird Satz 3 eingefügt, der feststellt, dass der WA über Einsprüche entscheidet.
Zuletzt ist ein Abs. 5 einzufügen, der die öffentliche Bekanntmachung der Entscheidungen über Einsprüche regelt und hierbei auf \S~10a verweist.
}

\begin{contract}
\Clause{title={Wählendenverzeichnis}}
Die Hochschulverwaltung erstellt auf Antrag ein Verzeichnis, das Familien- und Vornamen der Wahlberechtigten und die Matrikelnummer enthält (Wählendenverzeichnis).
Der Antrag ist an das Fachschaftenkollektiv zu richten.
Das Wählendenverzeichnis wird von der Wahlleitung bis spätestens zum 19. Tag vor dem ersten Wahltag übernommen.

Bei der Aufstellung und Handhabung des Wählendenverzeichnisses ist den Erfordernissen des Datenschutzes Rechnung zu tragen.
Insbesondere ist dafür Sorge zu tragen,
\change{dass Abschriften und Kopien vom Wählendenverzeichnis durch Dritte unterbleiben.}{%
dass das Wählendenverzeichnis nur berechtigten Personen ausgehändigt wird und dass Abschriften und Kopien vom Wählendenverzeichnis durch Dritte unterbleiben.}

Das Wählendenverzeichnis wird vor Ablauf der Frist für Einsprüche gegen das Wählendenverzeichnis \oldT{nach \S\S~11 Abs. 1, 12 Abs. 4 dieser Ordnung }an mindestens 3 Werktagen (Montag - Freitag) ausgelegt.

Einsprüche gegen die Richtigkeit des Wählendenverzeichnisses können beim Wahlausschuss bis zur gesetzten Frist schriftlich eingereicht werden.
Der Einspruch muss begründet sein.
Der Wahlausschuss entscheidet über Einsprüche und passt das Wählendenverzeichnis entsprechend an.

\change{Die Entscheidungen über Einsprüche sind gem. \S~10a dieser Ordnung öffentlich bekanntzumachen und sollen dem FSR zwecks Rücksprache mit der Verwaltung angezeigt werden,
um einem erneuten Auftreten der selben Mängel vorzubeugen.}{%
Die Entscheidungen über Einsprüche sind zu protokollieren und den betroffenen Personen mitzuteilen und sollen dem Fachschaftenkollektiv zwecks Rücksprache mit der Verwaltung angezeigt werden, um einem erneuten Auftreten der selben Mängel vorzubeugen.}
\end{contract}

\bemFr{\textbf{zu \S~13:} in Abs. 1 wird auf den neu geschaffenen \S~10a verwiesen}
\begin{contract}
\Clause{title={Wahlbekanntmachung}}
Die Wahlleitung macht die Wahl spätestens bis zum 18. Tag vor dem ersten Wahltag nach den Regeln des \S~10a dieser Ordnung bekannt.

Die Bekanntmachung muss mindestens enthalten:
\begin{enumerate}
\item Ort und Datum ihrer Veröffentlichung,
\item die Wahltage,
\item Ort und Zeit der Stimmabgabe,
\item die Bezeichnung des zu wählenden Organs,
\item die Zahl der zu wählenden Mitglieder,
\item die Frist, innerhalb derer Kandidaturen eingereicht werden können,
\item das für die Entgegennahme der Kandidaturen zuständige Organ,
\item eine Darstellung des Wahlsystems nach \S~7,
\item einen Hinweis darauf, dass nur wählen kann, wer im Wählendenverzeichnis eingetragen ist,
\item einen Hinweis auf Ort und Zeit der Auslegung des Wählendenverzeichnisses,
\item einen Hinweis auf die Einspruchsmöglichkeit gegen das Wählendenverzeichnis,
\item einen Hinweis auf die Möglichkeit eines Antrags auf Briefwahl, sowie die bei der Briefwahl zu beachtenden Fristen.
\end{enumerate}

Das Fachschaftenkollektiv stellt den Wahlausschüssen Musterdokumente für unter anderem Wahlbekanntmachung, Stimmzettel und Briefwahl zur Verfügung.
\end{contract}

\bemFr{\textbf{zu \S~14:} Es wurde diskutiert, \enquote{zurückweisen} bzgl. nicht fristgerechter Kandidaturen in Abs. 1 S. 2 in \enquote{ignorieren} zu ändern, damit die Wahlleitung nicht jeden Tag in ihr Postfach gucken muss.
Dahingehend besteht aber noch keine Einigkeit.}

\begin{contract}
\Clause{title={Kandidaturen}}
Die Kandidaturen sind innerhalb der Frist für die Einreichung von Kandidaturen, Briefwahlanträgen und Einsprüchen gegen das Wählendenverzeichnis beim Wahlausschuss einzureichen.
Nicht fristgerecht eingereichte Kandidaturen sind von der Wahlleitung zurückzuweisen.
\bemFr{bzw. zu ignorieren, siehe Anmerkung}

Eine Kandidatur muss mindestens enthalten:
\begin{enumerate}
\item Familienname(n) der kandidierenden Person,
\item \change{Vorname(n) der kandidierenden Person,}{alle Vornamen der kandidierenden Person, welche in die Matrikel eingetragen sind,}
\item \newT{der Name, wie er auf dem Stimmzettel stehen soll,}
\item ladungsfähige Anschrift der kandidierenden Person,
\item E-Mail-Adresse der kandidierenden Person,
\item Matrikelnummer der kandidierenden Person,
\item Bezeichnung der Wahl, für die die Kandidatur gelten soll,
\item Unterschrift der kandidierenden Person.
\end{enumerate}

\newT{Über die Zulässigkeit des Namens in Abs. 2 Nr. 3 entscheidet der Wahlausschuss.
Der Name in Abs. 2 Nr. 3 ist zu begründen, wenn es nicht um das Weglassen von Teilen der Namen nach Abs. 2 Punkt 1 und 2 geht.
Dieser ist anzunehmen, wenn der Name öffentlich bekannt und eindeutig ist.}

Die Wahlleitung gibt unverzüglich die Namen der gemäß \S~11 Abs. 3 und 4 als gültig 361 zugelassenen Kandidaturen öffentlich innerhalb der Studierendenschaft und an geeigneter Stelle im Internet bekannt.
Die Bekanntgabe enthält Ort und Datum ihrer Veröffentlichung und wird von der Wahlleitung unterschrieben.
\end{contract}

\bemFr{\textbf{zu \S~15:}
weitgehend unverändert, nur die Nr. 1 bis 3 mit Kleinschreibung zu Anfang aus orthographischen Gründen,
zudem in Nr. 3 \enquote{in einer zufällig gelosten Reihenfolge} wird ersetzt durch \enquote{festgelegten Reihenfolge} mit Verweis auf die Norm,
in der eben dies geregelt wird. Zudem in Nr. 2 \enquote{die vergeben werden dürfen}, vorher stand da \enquote{Stimmen, die vergeben darf}, was überhaupt keinen Sinn ergibt.
Außerdem wird in Nr. 2 nach dem Relativsatz ein „und“ eingefügt, um den kumulativen Charakter der Voraussetzungen deutlicher erkennbar zu machen.}

\begin{contract}
\Clause{title={Stimmzettel}}
Für die Herstellung der Stimmzettel ist die Wahlleitung zuständig.
Die Stimmzettel dürfen nicht voneinander zu unterscheiden sein.

Die Stimmzettel für die Wahlen enthalten:
\begin{enumerate}
\item die Bezeichnung der Wahl, für die sie gelten,
\item einen Hinweis auf die Anzahl der Stimmen, die vergeben werden dürfen, und
\item die Namen der Kandidierenden in einer zuvor \change{nach \S~11 Abs. 5 dieser Ordnung vom Wahlausschuss festgelegten Reihenfolge}{vom Wahlausschuss gelosten zufälligen Reihenfolge}.
\end{enumerate}
Bei dem Namen jeder Kandidatur wird ein ankreuzbares Feld platziert.

Wurden nicht mehr Kandidaturen zur Wahl zugelassen als Sitze zu besetzen sind, so wird die Bindung an die vorgeschlagenen Kandidaturen aufgehoben.
Der Stimmzettel enthält dann zusätzlich ein Freifeld, in das jede wahlberechtigte Person eingetragen werden kann, um die Stimme für sie abzugeben.
Das Wählendenverzeichnis liegt dann im Wahllokal zur Einsicht aus.

\Clause{title={Briefwahl}}
Auf schriftlichen Antrag hin können Wahlberechtigte ihre Stimme per Briefwahl abgeben.
Der Antrag ist zu begründen und muss Name, Anschrift, ggf. Versandanschrift und Matrikelnummer der den Antrag stellenden Person enthalten.

Erkennt die Wahlleitung die vorgebrachten Gründe an, ist der Antrag angenommen.
Andernfalls entscheidet der Wahlausschuss über den Antrag.
Anträge auf Briefwahl, die nach der Frist für die Einreichung von Kandidaturen, Briefwahlanträgen und Einsprüchen gegen das Wählendenverzeichnis eingehen, sind abweichend hiervon direkt von der Wahlleitung abzulehnen.

Falls im Antrag nicht vermerkt ist, dass die Unterlagen persönlich abgeholt werden sollen, sendet der Wahlausschuss die Briefwahlunterlagen postalisch zu.
Der Zeitpunkt ist nach Möglichkeit so zu wählen, dass die Antwort unter Berücksichtigung von Postlaufzeiten noch vor Ende der Wahl erwartet werden kann.

Die Briefwahlunterlagen enthalten:
\begin{enumerate}
\item den Stimmzettel,
\item den Wahlumschlag,
\item eine Versicherung, dass der Stimmzettel persönlich ausgefüllt wurde und die Folgen einer unrichtigen Versicherung, einer doppelten Stimmabgabe oder anderer Wahlfälschungen bekannt sind (Wahlschein),
\item den Wahlbrief-Umschlag.
\end{enumerate}

Der Stimmzettel ist zu falten und in den Wahlumschlag einzulegen, der danach zu verschließen ist.
Auf dem Stimmzettel oder dem Wahlumschlag dürfen keinerlei Angaben zur wählenden Person oder sonstige Angaben gemacht werden.
Der verschlossene Wahlumschlag ist zusammen mit dem Wahlschein in den Wahlbrief-Umschlag einzulegen, der wiederum verschlossen (Wahlbrief) und an den Wahlausschuss gesandt werden muss.
Sind diese Bedingungen nicht erfüllt, so ist der Stimmzettel ungültig.

Der Wahlbrief muss spätestens bis zu dem vom Wahlausschuss festgesetzten Ende der Wahl bei der Wahlleitung eingegangen sein.
Die Stimmabgabe ist vom Wahlausschuss nach Eingang des Wahlbriefs anhand des Wahlscheins zu prüfen und der Wahlumschlag in eine als Briefwahlurne bestimmte Urne einzuwerfen.
Die Abgabe der Briefwahlstimme ist im Urnenbuch festzuhalten und der Wahlschein als Anlage dem Urnenbuch beizufügen.

Sämtliche Personen deren Antrag auf Briefwahl angenommen wurde sind in einer gesonderten Liste zu erfassen, die den Wahlhelfenden an den Urnen mitzugeben ist.
\end{contract}

\bemFr{\textbf{zu \S~17:} unverändert bis auf den als Abs. 2 Satz 2 einzufügenden Satz, dass Wahlhelfende keine Kandidierenden sein dürfen, und den neu zu schaffenden Absatz 10, der die ohnehin zu jeder demokratischen Urnenwahl gehörende Bannmeile um die Wahlurne herum ausdrücklich normiert.}

\begin{contract}
\Clause{title={Wahlablauf und Urnenbuch}}
Die Wahlurnen sind vor Beginn der Wahl öffentlich durch die Wahlleitung zu versiegeln.
Dies wird im Urnenbuch dokumentiert.

Jede Wahlurne muss stets von mindestens zwei Wahlhelfenden besetzt sein, die für die ordnungsgemäße Durchführung der Wahl an dieser Urne verantwortlich sind.
Die Wahlhelfenden dürfen keine Kandidierenden sein.
Verlässt eine dieser Personen die Wahlurne, so wird bis zu ihrer Rückkehr der Wahlakt an dieser Urne durch Zwischensiegelung unterbrochen, falls dadurch weniger als 2 Wahlhelfende an der Urne verbleiben würden.

Die Wahlhelfenden prüfen die Wahlberechtigung jeder Person die wählen möchte und händigen die Stimmzettel aus.
Die Wahlhelfenden haben dafür Sorge zu tragen, dass die Wahl geheim erfolgt.
Der Stimmzettel muss unbeobachtet ausgefüllt werden.

Vor Einwurf des Stimmzettels in die Urne kann die wählende Person ihre Stimmabgabe korrigieren.
Ihr wird dann ein neuer Stimmzettel ausgehändigt, der alte Stimmzettel wird eingezogen und vernichtet.

Bei Einwurf des Stimmzettels in die Urne wird der Ausweis der wählenden Person von den Wahlhelfenden markiert.

An jeder Wahlurne wird die vom Wahlausschuss herausgegebene Liste der Kandidierenden ausgelegt.

Im Urnenbuch werden mit Zeitstempel dokumentiert:
\begin{enumerate}
\item Übergabe der Urne an die Wahlhelfenden,
\item Rückgabe der Urne an den Wahlausschuss,
\item jede Entsiegelung der Urne,
\item jede Versiegelung der Urne,
\item hinzukommende oder die Urne verlassende Wahlhelfende,
\item jeder wählende Person mit laufender Nummer, Name, Matrikelnummer und Unterschrift.
\end{enumerate}
Die Eintragungen nach 1-5 sind von den beteiligten Wahlausschussmitgliedern und Wahlhelfenden zu unterzeichnen.

Nach Beendigung eines jeden Wahltages hat die Wahlleitung für die bestmöglich gesicherte, versiegelte Aufbewahrung von Urnen und sonstigen Wahlunterlagen zu sorgen.

Ergeben sich bei der Feststellung der ordnungsgemäßen Versiegelung Unregelmäßigkeiten, so hat der Wahlausschuss die erforderlichen Maßnahmen zu treffen.

Werbung für einzelne \change{Kandidaten}{Kandidierenden} ist in Sicht- und Hörweite der Wahlurne und -kabine zu unterlassen und durch Wahlausschuss und Wahlhelfende zu unterbinden.
Allgemeine Werbung für die Teilnahme an der Wahl ist zulässig.
\end{contract}

\bemFr{\textbf{zu \S~18:} unverändert, nur in Abs. 4 wurde vor dem \enquote{wenn} ein Komma eingefügt.}
\begin{contract}
\Clause{title={Auszählung der Stimmen}}
Die Auszählung erfolgt öffentlich.

Die Auszählung der Stimmen wird durch die Wahlleitung, die weiteren Mitglieder des Wahlausschusses und weitere hierfür bestimmte Helfende,
die nicht Kandidierende sein dürfen, unverzüglich nach dem Ende der Wahl durchgeführt.

Die Urnen werden von der Wahlleitung öffentlich entsiegelt. Dies wird im Urnenbuch dokumentiert.

Ein Stimmzettel ist ungültig, wenn
\begin{enumerate}
\item auf ihm mehr als eine Stimme abgegeben wurde;
\item er außer der ordnungsgemäßen Stimmabgabe Zusätze enthält;
\item der Wille der wählenden Person nicht zweifelsfrei erkennbar ist;
\item sich ein Eintrag in einem Freifeld gemäß \S~15 Abs. 3 nicht eindeutig einer wählbaren Person zuordnen lässt;
\item ein nicht amtlicher Stimmzettel verwendet wurde;
\item die Vorgaben für die Stimmabgabe per Briefwahl nach \S~16 nicht eingehalten wurden.
\end{enumerate}
Im Zweifelsfall entscheidet der Wahlausschuss über die Gültigkeit von Stimmen.
\end{contract}

\bemFr{\textbf{zu \S~19:}
Absatz 5 wurde im Sinne des vorigen Entwurfs für eine FSWO-ÄO neu gefasst, lediglich mit Blick auf die Bekanntmachung auf \S~10a verwiesen und ein Rechtschreibfehler in \enquote{Bekannmachung} [sic] korrigiert.}

\begin{contract}
\Clause{title={Bekanntgabe des Wahlergebnisses}}
Die Bekanntgabe des Wahlergebnisses muss enthalten:
\begin{enumerate}
\item Ort und Datum ihrer Veröffentlichung;
\item den Namen des gewählten Organs;
\item die Zahl der Wahlberechtigten;
\item die Zahl der abgegebenen Stimmen;
\item die Zahl der ungültigen Stimmen;
\item die Zahl der gültigen Stimmen;
\item die Zahl der auf jede einzelne Person entfallenden gültigen Stimmen;
\item die Angabe darüber, welche Personen gewählt sind und welche nicht;
\item Ort und Zeit der konstituierenden Sitzung;
\item den Hinweis auf die Möglichkeit des Einspruchs gegen das Wahlergebnis, die mit Bekanntgabe des Wahlergebnisses beginnende Einspruchsfrist von vierzehn Tagen, die vorgeschriebene Form des Einspruchs, sowie den Wahlprüfungsausschuss der Fachschaftenkonferenz als zuständige Stelle.
\end{enumerate}

Die Wahlleitung benachrichtigt die gewählten Personen unverzüglich schriftlich oder per E- Mail von ihrer Wahl.

Die Gewählten erklären spätestens bis zu Beginn der konstituierenden Sitzung die Annahme oder Nichtannahme der Wahl. Andernfalls gilt die Wahl als nicht angenommen.

Mit der Annahme der Wahl verpflichtet sich die jeweilige Person, an den Sitzungen des Gremiums teilzunehmen.

Das Wahlergebnis ist von der Wahlleitung und den weiteren Wahlausschussmitgliedern zu unterzeichnen und dauerhaft aufzubewahren.
Es ist unverzüglich nach \S~10a dieser Ordnung sowie durch die Öffentlichkeitsbeauftragte auf der Bekanntmachungsplattform der Studierendenschaft zu veröffentlichen.
Eine Kopie des Wahlergebnisses ist samt eines Verweises auf die Bekanntmachung im Internet innerhalb von 14 Tagen dem Fachschaftenkollektiv zu übersenden.
\end{contract}

\bemFr{\textbf{zu \S~20:}
auch die konstituierende Sitzung der gewählten Gremien soll bereits virtuell stattfinden können, daher wird ein entsprechender Abs. 2 ergänzt, entsprechend dem Wortlaut des vorigen Entwurfs für eine ÄO, nur Satz 1 wurde um eine Wiederholung gekürzt und ein Rechtschreibfehler beseitigt.}

\begin{contract}
\Clause{title={Zusammentritt der Gremien}}
Die Wahlleitung hat das gewählte Gremium zu seiner konstituierenden Sitzung einzuberufen.
Die Sitzung findet frühestens am 5. Tag und spätestens am 14. Tag nach dem letzten Wahltag statt.
Die Wahlleitung leitet die Sitzung bis zur Wahl einer \oldT{oder eines }Vorsitzenden.

\oldT{Die Wahlleitung kann festlegen, dass die konstituierende Sitzung als virtuelle Sitzung in elektronischer Kommunikation stattfindet, solange höhere Rechtsquellen dies erlauben.
Sie stellt sicher, dass die Öffentlichkeit teilnehmen kann.}

\Clause{title={Wahl des FSR-Vorstands}}
Wird der Fachschaftsrat direkt gewählt, so wählt er auf seiner konstituierenden Sitzung aus seiner Mitte \oldT{einen Vorsitzenden bzw. }eine Vorsitzende, \oldT{einen stellvertretenden Vorsitzenden bzw. }eine stellvertretende Vorsitzende und \oldT{einen Finanzreferenten bzw. }eine Finanzreferentin.

Wird eine Fachschaftsvertretung gewählt, so wählt sie \change{den Vorsitz}{die Vorsitzende} des Fachschaftsrats, \change{dessen Stellvertretung}{deren Stellvertreterin}, \oldT{den Finanzreferenten bzw. }die Finanzreferentin und weitere Mitglieder mit der Mehrheit ihrer Mitglieder.
Die Abwahl \change{von Vorsitz, stellvertretendem Vorsitz oder Finanzreferenten bzw.}{der Vorsitzenden, stellvertretenden Vorsitzenden oder} Finanzreferentin ist nur durch die Wahl einer Nachfolge möglich.

Die Fachschaftssatzung kann ergänzende Regelungen vorsehen.
\end{contract}

\section{WAHLPRÜFUNG}
\begin{contract}
\Clause{title={Wahlprüfungsausschuss der Fachschaftenkonferenz (WPAF)}}
Die Fachschaftenkonferenz wählt 5 Studierende der RFWU Bonn in den Wahlprüfungsausschuss der Fachschaftenkonferenz;
diese dürfen nicht zugleich Mitglied eines Fachschaftswahlausschusses sein.
Dem WPAF dürfen maximal zwei Mitglieder derselben Fachschaft angehören.

Der Vorsitz des Fachschaftenkollektivs oder eine von ihm benannte Person sitzen dem WPAF vor.
Der WPAF-Vorsitz beruft die Sitzungen ein und leitet diese.
Er darf nicht zugleich Mitglied eines Fachschaftswahlausschusses sein.
Stimmrecht haben nur Ausschussmitglieder.

Der WPAF wird einmal jährlich zu Beginn des Sommersemesters gewählt.
\change{\S~15 FKGO gilt entsprechend.}{%
Das Nähere regelt die Geschäftsordnung der Fachschaftenkonferenz.}
\end{contract}

\bemFr{\textbf{zu \S~23:} Absatz sechs wurde im Sinne des vorigen Entwurfs einer ÄO neu gefasst und ebenso wortgetreu dem Entwurf folgend die Absätze 7 und 8 durch die Absätze 7 bis 12 neu ersetzt.}

\begin{contract}
\Clause{title={Wahlprüfung}}
Die Wahl ist mit der Bekanntmachung des Wahlergebnisses unbeschadet eines Wahlprüfungsverfahrens gültig.

Gegen die Gültigkeit der Wahl können Wahlberechtigte binnen 14 Tagen nach Bekanntmachung des Wahlergebnisses Einspruch erheben.
Der Einspruch ist unter Angabe der Gründe dem Wahlprüfungsausschuss der Fachschaftenkonferenz schriftlich oder per E-Mail einzureichen.

Nach einer Wahlprüfung entscheidet die Fachschaftenkonferenz nach Vorschlag des Wahlprüfungsausschusses der Fachschaftenkonferenz über die Gültigkeit der Wahl.

Wird die Feststellung des Wahlergebnisses für ungültig erachtet, so ist sie aufzuheben und eine erneute Feststellung anzuordnen.

Die Wahl ist ganz oder teilweise für ungültig zu erklären, wenn wesentliche Bestimmungen über die Wahlvorbereitung, die Sitzverteilung, das Wahlrecht, die Wählbarkeit oder das Wahlverfahren verletzt worden sind, es sei denn, dass dies sich nicht auf die Sitzverteilung ausgewirkt hat.

Sämtliche Wahlunterlagen sind nach Beendigung der Wahl für 90 Tage sicher aufzubewahren und auf Verlangen an den Wahlprüfungsausschuss der Fachschaftenkonferenz zu übergeben.
Wählendenverzeichnis, Stimmzettel und Kandidaturen sind nach Ablauf dieser Frist zu vernichten.
Protokolle des Wahlausschusses und das Wahlergebnis dürfen nicht vernichtet werden und sind mindestens 6 Jahre aufzubewahren.

Bei Zweifeln an der ordnungsgemäßen Durchführung einer Wahl muss das FSK innerhalb von 30 Tagen nach Bekanntmachung des Wahlergebnisses die Prüfung der Wahl durch den WPAF veranlassen.

Der WPAF kann Wahlen innerhalb von 30 Tagen nach Bekanntmachung des Wahlergebnisses auch unabhängig von Einsprüchen im Rahmen einer stichprobenartigen Prüfung prüfen.

Innerhalb von 7 Tagen nach den Prüfungsanforderungen nach Abs. 2, 7 oder 8 fordert der WPAF die Dokumente bei der Fachschaft an.
Die Anforderung enthält im Mindesten eine Liste aller angeforderten Dokumente, die Frist, innerhalb derer sie eingereicht werden müssen, und einen Verweis auf Abs. 11.

Innerhalb von 30 Tagen nach Eingang der Anforderung der Dokumente hat die Fachschaft die Dokumente an den WPAF zu übermitteln.

Innerhalb von 30 Tagen nach Eingang der vollständigen Dokumente hat der WPAF dem FSK eine Beschlussempfehlung vorzulegen.
Sofern die vollständigen Dokumente nicht eingereicht werden, hat die Prüfung anhand der bis dahin eingereichten Dokumente spätestens innerhalb von 30 Tagen nach Ablauf der Frist nach Abs. 10 zu erfolgen.

Das FSK führt eine öffentlich im Internet einsehbare Liste, in der nach dem aktuellsten Stand für jede Fachschaft die folgenden Informationen vermerkt sind:
\begin{enumerate}
\item Zeitpunkt des Antrags auf das Wählendenverzeichnis
\item Art des Wahlsystems (vgl. \S\S~2, 25, 26)
\item der Wahltermin
\item Zeitpunkt der Zusendung des Wahlergebnisses durch die Wahlleitung, sowie ein Verweis, wo dieses im Internet bekanntgemacht wurde
\item aktueller Status eventueller Wahlprüfungen
\end{enumerate}

\Clause{title={Wiederholung der Wahl, Sonderfälle}}
Wird keine gültige Kandidatur eingereicht, so setzt der Wahlausschuss einmalig eine Nachfrist von bis zu drei Tagen für die Einreichung von Kandidaturen.
Wird auch innerhalb dieser Nachfrist keine gültige Kandidatur eingereicht, wird einmalig das Wahlverfahren von den bestehenden Wahlorganen auf der Grundlage des bereits aufgestellten Wählendenverzeichnisses nach Maßgabe dieser Wahlordnung wiederholt (Wiederholungswahl).
Der Wahlausschuss bestimmt in diesem Fall unverzüglich den Termin für die Wiederholungswahl und die weiteren Termine und Fristen gemäß \S~13 Abs. 5.
Zwischen Festlegung des Termins der Wiederholungswahl und ihrem 1. Wahltag müssen mindestens 20 Tage liegen.

Wurde bei einer Wiederholungswahl nach Ablauf einer Nachfrist keine Kandidatur eingereicht, so enthält der Stimmzettel keine Kandidaturen, sondern lediglich das Freifeld gemäß \S~15 Abs. 3.

Wird im Wahlprüfungsverfahren die Wahl ganz oder teilweise für ungültig erklärt, so ist sie unverzüglich in dem in der Entscheidung bestimmten Umfang zu wiederholen.
Eine teilweise Wiederholung ist nur zulässig, wenn sie im selben Semester wie die zu wiederholende Wahl abgeschlossen werden kann.

Bis zum Abschluss der Wiederholungswahl sind die vorherigen Besetzungen von Fachschaftsrat und Fachschaftsvertretung kommissarisch im Amt.

Ist es nicht möglich, den beschlossenen Wahltermin einzuhalten, legt die FSV bzw. der FSR nach Rücksprache mit dem Fachschaftenkollektiv einen Ersatztermin fest und wählt ggf. Wahlausschussmitglieder und Wahlleitung nach.
Das bereits aufgestellte Wählendenverzeichnis behält seine Gültigkeit.
Zwischen dem Beschluss und dem neuen ersten Wahltag müssen mindestens 20 Tage liegen.

Sind die Organe einer Fachschaft nicht in der Lage, einen Wahltermin festzulegen oder Wahlorgane zu wählen, so kann die Fachschaftenkonferenz diese Aufgaben in Absprache mit den Studierenden der Fachschaft im Einzelfall übernehmen.
Dies gilt insbesondere bei der Neugründung von Fachschaften, sofern noch keine gewählten Fachschaftsorgane existieren.
\end{contract}

\section{ABWEICHENDE WAHLSYSTEME}
\begin{contract}
\Clause{title={Option der Wahlvollversammlung}}
Die Fachschaftssatzung kann festlegen, dass die Wahl bei bis zu 500 Wahlberechtigten im Rahmen einer Wahlvollversammlung durchgeführt wird.

Die Wahl in einer Wahlvollversammlung findet als Persönlichkeitswahl ohne Bindung an die vorgeschlagenen Kandidaturen statt.
Jedes Mitglied der Fachschaft ist wählbar.
Briefwahl ist möglich.
\S~24 Abs. 1 findet keine Anwendung.

Bis zum 25. Tag vor dem Wahltag legt der Wahlausschuss Ort und Zeit der Wahlvollversammlung fest.
Die Punkte 1, 4 und 5 in \S~13 Abs. 5 entfallen. Punkt 2 in \S~15 Abs. 2 entfällt.

Die Tagesordnung der Wahlvollversammlung ist in der Wahlbekanntmachung zu veröffentlichen.
Sie muss mindestens die folgenden Punkte umfassen:
\begin{enumerate}
\item Eröffnung,
\item Aufnahme weiterer Kandidaturen,
\item Vorstellung der Kandidaturen,
\item Wahl,
\item Auszählung und Verlesung des Wahlergebnisses,
\item Konstituierung des FSR / der FSV.
\end{enumerate}

Unter TOP 2 können wahlberechtigte Mitglieder der Fachschaft ihre Bereitschaft zur Kandidatur erklären, die nicht bereits eine Kandidatur eingereicht haben.
Dies gilt nicht für die Wahlleitung und die Mitglieder des Wahlausschusses.

Unter TOP 3 können die anwesenden Kandidierenden sich persönlich vorstellen.
Abwesende Kandidierende können durch die Versammlungsleitung eine Vorstellung verlesen lassen.
Die Reihenfolge wird ausgelost.
Die maximale Dauer für eine einzelne Vorstellung 638 beschließt der Wahlausschuss vor Beginn der Wahlvollversammlung.

Die Wahl erfolgt unter TOP 4 gemäß \S~17.
Der Stimmzettel enthält ausschließlich ein Freifeld.
Die Wahl endet, sobald keine der im Versammlungsraum anwesenden Wahlberechtigten mehr ihre Stimme abgeben möchte.

Unter TOP 5 wird die Wahl öffentlich ausgezählt, das offizielle Ergebnis durch den Wahlausschuss festgestellt und im Anschluss verlesen.

Das gewählte Gremium konstituiert sich abweichend von \S~20 direkt nach Bekanntgabe des Wahlergebnisses unter TOP 6.
Abweichend von \S~19 Abs. 2 haben die Gewählten ab Bekanntgabe des Wahlergebnisses 7 Kalendertage Zeit, die Wahl anzunehmen.
Andernfalls gilt die Wahl als nicht angenommen.

Die übrigen Vorschriften dieser Wahlordnung sind entsprechend anzuwenden.
\end{contract}

\bemFr{\textbf{Achtung, Obacht, Aufgemerkt!}

Die folgende Norm ist noch umstritten – der vorige Entwurf für eine Änderungsordnung sah eine Reformierung vor, die sich als im SP nicht mehrheitsfähig erwiesen hat.
Im SGO wurde ein Kompromiss ausgearbeitet, mit dem alle Anwesenden einverstanden waren – diesen hat Felix Blanke ausformuliert und stellt ihn unabhängig von diesem Dokument vor.

Abgedruckt ist daher hier die bislang geltende Fassung.}


\begin{contract}
\Clause{title={Option der Verhältniswahl}}
Die Fachschaftssatzung kann festlegen, dass die Wahl abweichend von \S~4 Abs. 2 nicht als Persönlichkeitswahl, sondern als personalisierte Verhältniswahl durchgeführt wird.

%Neben dem in Abs. 1 genannten Fall wird die Wahl ebenfalls als personalisierte Verhältniswahl durchgeführt, wenn
%\begin{enumerate}
%\item bei weniger als 11 zu wählenden Personen 5 Wahlberechtigte
%\item bei 11 zu wählenden Personen 6 Wahlberechtigte
%\item bei 15 zu wählenden Personen 8 Wahlberechtigte
%\item bei 19 zu wählenden Personen 10 Wahlberechtigte
%\end{enumerate}
%dies vor der Wahl eines Wahlausschusses für diese Wahl verlangen.
%Das Verlangen ist der FSV bzw. falls keine existiert dem FSR einzureichen.
%Es muss in zeitlichem Zusammenhang zur Wahl stehen.}
\bemFe{Dieser Absatz wurde neu gefasst.}
\begingroup
\makeatletter
  \ifFK@draft\color{Green}\fi
\makeatother
Neben dem in Abs. 1 genannten Fall wird die Wahl ebenfalls als
personalisierte Verhältniswahl durchgeführt, wenn
\begin{enumerate}
\item bei weniger als 11 zu wählenden Personen 5 Wahlberechtigte
\item bei 11 zu wählenden Personen 6 Wahlberechtigte
\item bei 15 zu wählenden Personen 8 Wahlberechtigte
\item bei 19 zu wählenden Personen 10 Wahlberechtigte
\end{enumerate}
dies 25 Tage vor dem 1. Wahltag für diese Wahl verlangen. Das Verlangen ist der Wahlleitung einzureichen und von dieser zu prüfen.
\endgroup

Im Falle einer personalisierten Verhältniswahl muss jeder Wahlvorschlag einen Listennamen und mindestens eine Kandidatur in einer erkennbaren Reihenfolge enthalten.
Eine Person kann nicht auf mehreren Wahlvorschlägen kandidieren.

\newT{Wurde nur genau ein Wahlvorschlag oder nicht mehr Kandidaturen als Sitze zu besetzen sind zur Wahl zugelassen, so wird statt der Verhältniswahl eine Persönlichkeitswahl durchgeführt.}

Der Stimmzettel enthält die Wahlvorschläge in einer vom Wahlausschuss gelosten zufälligen Reihenfolge, für jeden Wahlvorschlag unter dem Listennamen die Kandidaturen in der Reihenfolge aus dem Wahlvorschlag.
Angekreuzt werden kann ein gesamter Wahlvorschlag oder eine einzelne Kandidatur.
\oldT{Die Regelung zum Freifeld (\S~15 Abs. 3) findet keine Anwendung.}

Die Verteilung der Sitze auf die Wahlvorschläge erfolgt nach dem Sainte-Laguë-Verfahren auf Grundlage der Gesamtzahl der Stimmen, die für einen Wahlvorschlag und seine Kandidaturen abgegeben wurden.
Bei Höchstzahlgleichheit wird durch die Wahlleitung ausgelost, welchem Wahlvorschlag der Sitz zugeteilt wird.
Innerhalb eines Wahlvorschlags werden die Sitze absteigend nach der Anzahl der erhaltenen Personenstimmen verteilt.
Bei Stimmengleichheit wird durch die Wahlleitung gelost.
Beim Ausscheiden eines Mitglieds rücken solange die Nächstplatzierten des selben Wahlvorschlags nach, bis der Wahlvorschlag erschöpft ist.
Ist der Wahlvorschlag erschöpft, bleibt der Sitz unbesetzt.
Abweichend von der Regelung in \S~8 Abs. 2 und 4 ist keine Mindeststimmzahl erforderlich.

Die Bekanntgabe des Wahlergebnisses gemäß \S~19 enthält zusätzlich die Zahl der auf jeden Wahlvorschlag entfallenden Stimmen und Sitze.

Wurde bei einer Wiederholungswahl nach Ablauf einer Nachfrist keine Kandidatur eingereicht (\S~24 Abs. 2), so wird statt der Verhältniswahl eine Persönlichkeitswahl durchgeführt, bei der der Stimmzettel abweichend von Abs. 3 lediglich das Freifeld gemäß \S~15 Abs. 3 enthält.

Die übrigen Vorschriften dieser Wahlordnung sind entsprechend anzuwenden.

Eine Festlegung nach Abs. 1 kann innerhalb der letzten 30 Tage vor Beginn einer Wahl nicht mehr mit Wirkung für diese Wahl hinzugefügt oder gestrichen werden.
Die Hinzufügung oder Streichung einer solchen Festlegung durch eine FSVV erfolgt mit der Mehrheit der anwesenden Fachschaftsmitglieder; anderslautende Mehrheitserfordernisse der Fachschaftssatzung sind unbeachtlich.
Falls eine solche Festlegung durch eine FSVV hinzugefügt oder gestrichen wird, kann die FSV bis nach Abschluss der nächsten Wahl keinen davon abweichenden Beschluss fassen. 4 Abs. 2 bleibt unberührt.

Die Durchführung einer Verhältniswahl gemäß \S~26 hat stets Vorrang vor der Durchführung einer Persönlichkeitswahl in einer Wahlvollversammlung gemäß \S~25.
\end{contract}

\section{SCHLUSSBESTIMMUNGEN}
\bemFr{Anmerkungen zu \S~27: Abs. 2 wird geändert, um den Wahlausschüssen ab dem Zeitpunkt ihrer Bestellung Rechtssicherheit zu geben, sie sollen nicht durch politische Entscheidungen und künftige Reformbestrebungen in ihrer Arbeit behindert werden.
Zudem wurde Abs. 3 dem neuen Publikationsverfahren angepasst. Der neu geschaffene Abs. 4 schafft Rechtssicherheit für Fälle der vollständigen Neufassung (statt Änderung mittels ÄO) dieser Ordnung.}

\begin{contract}
\Clause{title={Änderung der Wahlordnung}}
Diese Wahlordnung kann nur auf Vorschlag der Fachschaftenkonferenz vom Studierendenparlament mit der Mehrheit seiner satzungsgemäßen Mitglieder geändert werden.
Als eine Änderung ist sowohl eine Änderung des Wortlautes dieser Wahlordnung als auch die Ergänzung und Aufhebung von Bestimmungen anzusehen.
\begin{addmargin}{0pt}
\oldT{(2) ALTE FASSUNG Die Wahlordnung kann innerhalb der letzten 30 Tage vor Beginn 715 einer Wahl nicht mehr mit Wirkung für diese Wahl geändert werden.}
\end{addmargin}

Änderungen der Wahlordnung sind nur auf Wahlen anzuwenden, deren Wahlausschuss nach Inkrafttreten der Änderungen bestellt wurde.

Die Änderungen treten mit ihrer Veröffentlichung durch die Öffentlichkeitsbeauftragte der Studierendenschaft \newT{auf der Bekanntmachungsplattform der Studierendenschaft} in Kraft.

Die Absätze 1 bis 3 gelten auch für die Errichtung einer neuen Ordnung mit identischem Geltungsbereich, die diese Ordnung ersetzt.
\end{contract}

\bemFr{\textbf{Anmerkungen zu \S~28:}
Die amtliche Überschrift der Norm wird um \enquote{und Geltungsdauer} ergänzt, da dies den Regelungsgegenstand präziser umschreibt.
Absatz 1 wird dem neuen Publikationsverfahren angepasst.
In Absatz 2 wird das Datum derjenigen Neufassung genannt, die durch die bislang geltende Ordnung ersetzt wurde –
das Datum ist dahingehend angepasst, dass hier die durch diese Neufassung zu verdrängende, bislang gültige Fassung benannt werden muss.

In Absatz 2 Satz 2 wird die 30-Tage-Regelung wie in \S~27 II durch eine Regelung ersetzt, die Wahlausschüssen von kurz nach Inkrafttreten dieser Ordnung durchzuführenden Wahlen ein erhöhtes Maß an Rechtssicherheit zubilligt. Der neu zu schaffende Absatz 3 dient der Klarstellung der Normenhierarchie.}

\begin{contract}
\Clause{title={Inkrafttreten und Geltungsdauer}}
Diese Wahlordnung tritt ab dem Zeitpunkt ihrer Veröffentlichung durch die Öffentlichkeitsbeauftragte der Studierendenschaft \newT{auf der Bekanntmachungsplattform der Studierendenschaft} in Kraft.

Mit Inkrafttreten dieser Wahlordnung tritt die Wahlordnung für die Wahlen der Fachschaftsvertretungen und Fachschaftsräte in der Fassung der Neufassung vom 16. Mai 2017 außer Kraft.
Auf Wahlen, deren Wahlausschuss bereits vor Inkrafttreten dieser Ordnung bestellt wurde, findet die Wahlordnung für die Wahlen der Fachschaftsvertretungen und Fachschaftsräte der Rheinischen Friedrich-Wilhelms-Universität Bonn in der Fassung der Neufassung vom 16. Mai 2017 Anwendung.

\change{Die Satzung der Studierendenschaft und höherrangiges Recht brechen diese Ordnung.
Diese Ordnung bricht die Satzung der Fachschaft.}{Die Satzung der Studierendenschaft sowie höherrangiges Recht gehen dieser Ordnung vor. Diese Ordnung geht der Fachschaftssatzung vor.}
\end{contract}


\vspace{2em}
{\itshape%
Ausgefertigt aufgrund des Beschlusses der Fachschaftenkonferenz am dd.~November~2020 sowie des Beschlusses des Studierendenparlaments am dd.~November~2020.

Bonn, der dd.~November~2020}
\vspace{1em}
\begin{center}
\textsc{Name Einfügen}\\
Vorsitzender der \textsc{Amt Einfügen}
\end{center}


\end{document}