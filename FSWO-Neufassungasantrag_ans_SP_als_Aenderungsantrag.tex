% !TeX program = lualatex
\documentclass[DIV=12, parskip=half, fontsize=12pt, a4paper]{scrartcl}

%\usepackage[utf8]{inputenc}
%\usepackage[T1]{fontenc}    % Font-Encoding
\usepackage{lmodern}        % LModern font
\usepackage[ngerman]{babel} % deutsche Lokalisierung
%\usepackage{graphicx}       % Einbindung von Bildern
\usepackage{pdfpages}


\usepackage{lineno}         % Nummerierung der Zeilen
%\usepackage{csquotes}       % bessere Anführungszeichen
%\usepackage{eurosym}        % Euro-Zeichen setzen
%\usepackage[normalem]{ulem} % Durchstreichen von Text-Passagen
%\usepackage[shortlabels]{enumitem} % bessere Aufzählungen mit Kurzschreibweise der Labels

\usepackage{pdfpages}       % erlaubt das Einbinden von Seiten anderer PDF-Dateien


\usepackage{draftwatermark}
\SetWatermarkText{Entwurf}


\title{Änderungsantrag zum Antrag zur Änderung der Wahlordnung für die Wahlen der Fachschaftsvertretungen und Fachschaftsräte (FSWO)}
\author{Fachschaftenkonferenz}
\date{\today}

\begin{document}

	\maketitle
	Das SP möge auf Vorschlag der Fachschaftenkonferenz beschließen:

	\begin{linenumbers}
		Der Antrag zur Änderung der Wahlordnung für die Wahlen der Fachschaftsvertretungen und Fachschaftsräte (FSWO) vom 7. Oktober 2019 (im 41. SP als Antrag 48/41 in 1. Lesung in der 11. ordentlichen Sitzung am 20.11.2019 diskutiert und im 42. SP erneut in 1. Lesung in der 1. ordentlichen Sitzung am 26.02.2020 diskutiert) wird wie folgt neugefasst:

		\begin{center}\bfseries\LARGE Antrag zur Neufassung der Fachschaftswahlordnung
		\end{center}

		Das SP möge auf Vorschlag der Fachschaftenkonferenz beschließen:

		Die Wahlordnung für die Wahlen der Fachschaftsvertretungen und Fachschaftsräte der Studierendenschaft der Rheinischen Friedrich-Wilhelms-Universität Bonn vom 16. Mai 2017 wird durch die beigefügte Wahlordnung geändert und neugefasst.

		Das  SP-Präsidium und das Fachschaftenkolletiv  werden  damit  beauftragt,  diesen Beschluss unverzüglich auszufertigen und die neugefasste Wahlordnung zur Veröffentlichung durch die Öffentlichkeitsbeauftragte zu bringen.
	\end{linenumbers}

	\vspace{1em}
	\textit{Ausgefertigt aufgrund des Beschlusses der Fachschaftenkonferenz am dd. November 2020.}

	Bonn, den \today \\
	Christoph Liedel \\
	{\scriptsize Vorsitzender des Fachschaftenkollektivs, Fachschaftenreferent}

	\clearpage
	\includepdf[pages=-]{FSWO-Entwurf-ohne-Anmerkungen}
\end{document}
